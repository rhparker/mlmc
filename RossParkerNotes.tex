\documentclass[]{article}

\usepackage{amsmath}
\usepackage{enumerate}
\usepackage{amssymb}                
\usepackage{amsmath}                
\usepackage{amsfonts}
\usepackage{amsthm}

\usepackage{mathtools}
\usepackage{cool}
\usepackage{graphicx}

\DeclarePairedDelimiter\abs{\lvert}{\rvert}%
\DeclarePairedDelimiter\norm{\lVert}{\rVert}%

\newtheorem{theorem}{Theorem}[section]
\newtheorem{corollary}{Corollary}[theorem]
\newtheorem{proposition}[theorem]{Proposition}
\newtheorem{lemma}[theorem]{Lemma}

\theoremstyle{definition}
\newtheorem{definition}[theorem]{Definition}

\theoremstyle{assumption}
\newtheorem{assumption}{Assumption}

\theoremstyle{remark}
\newtheorem*{question}{Question}
\newtheorem*{observation}{Observation}
\newtheorem*{remark}{Remark}

\setlength{\parindent}{0cm}

\title{Title}
\author{Author}
\date{Today}

\begin{document}

\section{}

\subsection{Preliminaries}
Consider the reflected SDE on an open domain $U$ in $\mathbb{R}^d$:
\[
dX_t = b(X_t, t)dt + \sigma(X_t, t)dW_t + \nu(X_t)dL_t
\]
Where $W_t = (W_1(t), ..., W_d(t) )$ is a $d$-dimensional Brownian motion, $\nu$ is an oblique reflection vector on the boundary of $U$, and $L_t$ is the local time on the boundary. We will consider the solution on the closed interval $[0, T]$. For simplicity of notation, we will take $b$ and $\sigma$ to be independent of $t$, in which case the SDE reduces to:
\begin{equation}
dX_t = b(X_t)dt + \sigma(X_t)dW_t + \nu(X_t)dL_t
\end{equation}
where $b: \mathbb{R}^d \rightarrow \mathbb{R}^d$ and $\sigma: \mathbb{R}^d \rightarrow \mathbb{R}^d \times  \mathbb{R}^d$. In coordinate form, we have for $i = 1, ..., d$:
\begin{equation}
dX_i(t) = b_i(X_t)dt + \sum_{j = 1}^d \sigma_{ij}(X_t)dW_j(t) + \nu_i(X_t)dL_i(t)
\end{equation} 
We are trying to prove the following proposition (variation of Lemma 5.1 on p.9 of the Giles paper):
\begin{proposition}For all $t \leq T$ there exist constants $c$ and $K(T)$ such that:
\begin{equation}
\mathbb{P}\left(\sup_{[0, t]} \norm{X_t} \geq \eta \right) \leq K(T)exp\left( -\frac{c \eta^2}{t}  \right)
\end{equation}
\end{proposition}
We make the following standard assumptions on the coefficient functions $b$ and $\sigma$:

\begin{assumption}The coefficient functions are Lipschitz and satisfy a growth condition, i.e. there exist constants $K_1$ and $K_2$ such that for all $x$ and $y$: 

\begin{equation}
\abs{b(x) - b(y)} +  \norm{\sigma(x) - \sigma(y)} \leq K_1 \abs{x - y}
\end{equation}
%
\begin{equation}
\abs{b(x)}^2 + \norm{\sigma(x)}^2 \leq K_2(1 + \abs{x}^2 )
\end{equation}
In addition, for simplicity we will assume that the initial condition is $X(0) = 0$ a.s. \\ \\
\end{assumption}
Now write the SDE in integrated form. Recalling that we are starting at 0 a.s.:
\[
X(t) = \int_0^t b(X_s)ds + \int_0^t  \sigma(X_s)dW_s + \int_0^t \nu(X_s)dL_s
\]
X(t) is continuous and is confined to the region $U$. The final term on the RHS is nonnegative and increases only on the boundary of $U$. Thus X(t) is the unique solution to the Skorokhod problem for:
\[
H(t) =  \int_0^t b(X_s)ds + \int_0^t  \sigma(X_s)dW_s 
\]
Letting $\Gamma: C[0, T] \rightarrow C[0, T]$ be the Skorokhod mapping, we have $X = \Gamma[H]$. Substituting this above, we obtain the functional SDE for H:
\begin{equation}
H(t) =  \int_0^t b(\Gamma(H)(s))ds + \int_0^t  \sigma(\Gamma(H)(s))dW_s 
\end{equation}
In component form, this is:
\begin{equation}
H_i(t) =  \underbrace{ \int_0^t b_i(\Gamma(H)(s))ds }_{A_i(t)}+ \underbrace{\sum_{j=1}^d \int_0^t  \sigma_{ij}(\Gamma(H)(s))dW_s }_{M_i(t)}
\end{equation}
In this functional SDE, $\Gamma(H)(t)$ depends on the entire past up to time $t$, i.e. depends on $H_s, 0 \leq s \leq t$. We will rewrite this equation to make its functional nature explicit. (This is based on \cite[Chapter 5]{Mao97} ). Before we do that, so that the domain of our coefficient functions will be consistent, we define the past history of a process $H$ at time $t$ to be the process $\tilde{H}_t \in C[-T, 0]$ given by:
\begin{equation}
\tilde{H}_t(r) = \{H_i(t + r), r \in [-T, 0] \}
\end{equation}
where we take $H_i(s) = 0$ for $s < 0$. Then we can define the new coefficient functions by:
\begin{align*}
\tilde{b}&: C[0, T] \rightarrow \mathbb{R}^d && \tilde{b}_i(X_t) = b( \Gamma(X_t)(T)) \\
\tilde{\sigma}&: C[0, T] \rightarrow \mathbb{R}^d \times \mathbb{R}^d && \tilde{\sigma}(X_t) = \sigma(\Gamma(X_t)(T))
\end{align*}
Which lets us rewrite the system as:
\begin{equation}
H_i(t) =  \underbrace{ \int_0^t \tilde{b}_i(\tilde{H}_s)}_{A_i(t)}+ \underbrace{\sum_{j=1}^d \int_0^t  \tilde{\sigma}_{ij}(\tilde{H}_s)dW_s }_{M_i(t)}
\end{equation}

We will consider two cases for the reflecting boundary conditions. First, in the case where U is a convex polygon with nonempty interior and the reflection directions are normal, there is a unique solution to the Skorokhod problem, and the Skorokhod map $\Gamma$ is Lipschitz continuous \cite{Dupuis91}. This can be extended to oblique reflections when U is a convex polygon and the reflection directions are constant on each face and satisfy conditions (look these up!) \cite{Dupuis91}. In the case of normal reflection inside a general smooth, open domain, Lions and Sznitman \cite[Theorem 1.1]{Lions84} showed that there is again a unique solution to the Skorokhod problem, but that the Skohokhod map $\Gamma$ is only Holder continuous of order 1/2 on $[0, T]$.

\subsection{Lipschitz continuity of Skorokhod map}
First, consider the case where we have a unique solution to the Skorokhod problem, and the map $\Gamma$ is Lipschitz continuous. Recall that the Skorohod map is a map $\Gamma: C[0, T] \rightarrow C[0, T]$, where we are using the $L^\infty$ norm on $C[0, T]$. Let $C_T$ be the Lipschitz constant. Then since $\Gamma(H) = X$, we have:
\[
\sup_{[0, T]} \abs{ X(s) } \leq C_T \sup_{[0, T]} \abs{ (H)(s) }
\]
Since we can``cut off'' any continuous function at $t$ by sending it linearly to 0 on $[t, T]$, this relation holds with the same Lipschitz constant for all $t \in [0, T]$:
\[
\sup_{[0, t]} \abs{ X(s) } \leq C_T \sup_{[0, t]} \abs{ (H)(s) }
\]
Then we have the following set of inequalities concerning the proposition we are trying to prove:
\begin{align*}
\mathbb{P}\left(\sup_{[0, t]} \norm{X_s} \geq \eta \right) &\leq  \mathbb{P}\left(\sup_{[0, t]} C_T\norm{H_s} \geq \eta \right) \\
&= \mathbb{P}\left(\sup_{[0, t]} \norm{H_s} \geq \frac{\eta}{C_T} \right)
\end{align*}
Since all (reasonable) norms on $\mathbb{R}^d$ are equivalent, we will choose the $l^1$ norm.
\begin{align*}
\mathbb{P}\left(\sup_{[0, t]} \norm{X_t} \geq \eta \right) &\leq \mathbb{P}\left(\sup_{[0, t]} \sum_{i=i}^d \abs{H_i(s)} \geq \frac{\eta}{C_T} \right) \\
&\leq \sum_{i=i}^d \mathbb{P}\left(\sup_{[0, t]}  \abs{H_i(s)} \geq \frac{\eta}{C_T d} \right) \\
&\leq \sum_{i=i}^d \mathbb{P}\left(\sup_{[0, t]}  \abs{A_i(s)} + \sup_{[0, t]}  \abs{M_i(s)} \geq \frac{\eta}{C_T d} \right) 
\end{align*}
where $A_i(s)$ and $M_i(s)$ are as defined above. Then we have:
\begin{align*}
\mathbb{P}\left(\sup_{[0, t]} \norm{X_t} \geq \eta \right) &\leq \sum_{i=i}^d \left[ \mathbb{P}\left(\sup_{[0, t]}  \abs{A_i(s)}  \geq \frac{\eta}{2 C_T d} \right) + \mathbb{P}\left(\sup_{[0, t]}  \abs{M_i(s)}  \geq \frac{\eta}{2 C_T d} \right) \right] \\
&\leq \sum_{i=i}^d \left[ \mathbb{P}\left(\sup_{[0, t]}  A_i(s)  \geq \frac{\eta}{2 C_T d} \right) + \mathbb{P}\left(\sup_{[0, t]}  M_i(s) \geq \frac{\eta}{2 C_T d} \right) \right] 
\end{align*}
where in the last line we have removed the absolute value since $A_i(0) = 0$ and $M_i(0) = 0$. We will then bound the two terms on the RHS individually.

\subsection{Ordinary Integral}

\subsection{Stochastic Integral}

% references
\begin{thebibliography}{9}

\bibitem{Dupuis91}Dupuis, P. and Ishii, H. (1991) On Lipschitz continuity of the solution
mapping to the Skorokhod problem, with applications, Stochastics and Stochastic Reports,
35:1, 31-62.

\bibitem{Lions84} Lions, P. L. and Sznitman, A. S. (1984), Stochastic differential equations with reflecting boundary conditions. Comm. Pure Appl. Math., 37: 511–537.

\bibitem{Mao97}Mao, X (1997). Stochastic Differential Equations and Applications, Horwood

\end{thebibliography}

\end{document}