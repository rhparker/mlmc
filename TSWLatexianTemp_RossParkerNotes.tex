\documentclass[]{article}

\usepackage{amsmath}
\usepackage{enumerate}
\usepackage{amssymb}                
\usepackage{amsmath}                
\usepackage{amsfonts}
\usepackage{amsthm}

\usepackage{mathtools}
\usepackage{cool}
\usepackage{graphicx}

\DeclarePairedDelimiter\abs{\lvert}{\rvert}%
\DeclarePairedDelimiter\norm{\lVert}{\rVert}%

\newtheorem{theorem}{Theorem}[section]
\newtheorem{corollary}{Corollary}[theorem]
\newtheorem{proposition}[theorem]{Proposition}
\newtheorem{lemma}[theorem]{Lemma}

\theoremstyle{definition}
\newtheorem{definition}[theorem]{Definition}

\theoremstyle{assumption}
\newtheorem{assumption}{Assumption}

\theoremstyle{remark}
\newtheorem*{question}{Question}
\newtheorem*{observation}{Observation}
\newtheorem*{remark}{Remark}

\setlength{\parindent}{0cm}

\title{Title}
\author{Author}
\date{Today}

\begin{document}

\section{}

Consider the reflected SDE on an open domain $U$ in $\mathbb{R}^d$:
\[
dX_t = b(X_t, t)dt + \sigma(X_t, t)dW_t + \nu(X_t)dL_t
\]
Where $W_t = (W_1(t), ..., W_d(t) )$ is a $d$-dimensional Brownian motion, $\nu$ is an oblique reflection vector on the boundary of $U$, and $L_t$ is the local time on the boundary. We will consider the solution on the closed interval $[0, T]$. For simplicity of notation, we will take $b$ and $\sigma$ to be independent of $t$, in which case the SDE reduces to:
\begin{equation}
dX_t = b(X_t)dt + \sigma(X_t)dW_t + \nu(X_t)dL_t
\end{equation}
where $b: \mathbb{R}^d \rightarrow \mathbb{R}^d$ and $\sigma: \mathbb{R}^d \rightarrow \mathbb{R}^d \times  \mathbb{R}^d$. In coordinate form, we have for $i = 1, ..., d$:
\begin{equation}
dX_i(t) = b_i(X_t)dt + \sum_{j = 1}^d \sigma_{ij}(X_t)dW_j(t) + \nu_i(X_t)dL_i(t)
\end{equation}
We make the following standard assumptions on the coefficient functions $b$ and $\sigma$:

\begin{assumption}The coefficient functions are Lipschitz and satisfy a growth condition, i.e. there exist constants $K_1$ and $K_2$ such that for all $x$ and $y$: 

\begin{equation}
\abs{b(x) - b(y)} +  \norm{\sigma(x) - \sigma(y)} \leq K_1 \abs{x - y}
\end{equation}
%
\begin{equation}
\abs{b(x)}^2 + \norm{\sigma(x)}^2 \leq K_2(1 + \abs{x}^2 )
\end{equation}
In addition, for simplicity we will assume that the initial condition is $X(0) = 0$ a.s. \\ \\
\end{assumption}
Now write the SDE in integrated form. Recalling that we are starting at 0 a.s.:
\[
X(t) = \int_0^t b(X_t)dt + \int_0^t  \sigma(X_t)dW_t + \int_0^t \nu(X_t)dL_t
\]
X(t) is continuous and is confined to the region $U$. The final term on the RHS is nonnegative and increases only on the boundary of $U$. Thus X(t) is the unique solution to the Skorohod problem for:
\[
H(t) =  \int_0^t b(X_t)dt + \int_0^t  \sigma(X_t)dW_t 
\]
Letting $\Gamma: C[0, T] \rightarrow C[0, T]$ be the Skorohod mapping, we have $X = \Gamma[H]$. Substituting this above, we obtain the functional SDE for H:
\begin{equation}
H(t) =  \int_0^t b(\Gamma(H)(t))dt + \int_0^t  \sigma(\Gamma(H)(t))dW_t 
\end{equation}

We will consider two cases for the reflecting boundary conditions. In the case of normal reflection inside a smooth, open domain, Lions and Sznitman \cite{Lions} showed that there is a unique solution to the Skorokhod problem, and that the Skohokhod map $\Lambda$ is Holder continuous of order 1/2 on $[0, T]$.

% references
\begin{thebibliography}{9}

\bibitem{Lions}
Lions, P. L. and Sznitman, A. S. (1984), Stochastic differential equations with reflecting boundary conditions. Comm. Pure Appl. Math., 37: 511–537. doi: 10.1002/cpa.3160370408

\end{thebibliography}

\end{document}